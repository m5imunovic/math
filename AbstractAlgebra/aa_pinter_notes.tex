\documentclass[a4paper,12pt]{article}
\usepackage{amsmath}
\usepackage{amsthm}
\usepackage{amsfonts}
\usepackage{amssymb}
\usepackage[margin=1in]{geometry}
%\usepackage{showframe}
\title{Notes on Abstract Algebra by C.C. Pinter}
\author{M. Simunovic}

\theoremstyle{definition}
\newtheorem{definition}{}[section]
\theoremstyle{axiom}
\newtheorem{axiom}{G}[]
\theoremstyle{theorem}
\newtheorem{theorem}{Theorem}[section]
\theoremstyle{lemma}
\newtheorem{lemma}{Lemma}[section]
\begin{document}
\maketitle

\section{Notes to me}
\begin{itemize}
        \item Set can be anything, not limited to numbers.
        \item Always remember that sets might contain infinite number of members.
\end{itemize}
\section{Operations}
An operation $*$ on $A$ is a rule which assigns to each ordered pair $(a,b)$ of elements of A exactly one element $a * b$ in A.
\begin{definition}{}
        \textnormal{$a * b$ is defined for \textbf{every} ordered pair $(a, b)$ of elements of $A$.}
\end{definition}
\begin{definition}{}
        \textnormal{$a * b$ must be \textbf{uniquely} defined.}
\end{definition}
\begin{definition}{}
        \textnormal{If $a$ and $b$ are in $A$, $a * b$ must be in $A$.}
\end{definition}
\begin{definition}{}
        \textnormal{Operation is \textit{commutative} if $a * b = b * a$ for any $a$ and $b$ in $A$.}
\end{definition}
\begin{definition}{}
        \textnormal{Operation is \textit{associative} if $(a * b) * c = a * (b * c)$ for any three elements $a$, $b$ and $c$ in $A$.}
\end{definition}
\begin{definition}{}
        Element $e \in A$ with the property $e * a = a$ and $a * e = a$ for every element $a \in A$  is called identity element.
\end{definition}
\begin{definition}{}
        \textnormal{Element $x \in A$ with the property $x * a = e$ and $a * x = e$ for any element $a \in A$ is called inverse of $a$.}
\end{definition}

\section{The Definition of Groups}
A \textbf{group} $\langle G, * \rangle$ is \textit{a set $G$ with an operation $*$} that satisfies the axioms:
\begin{axiom}
$*$ is associative
\end{axiom}
\begin{axiom}
There is an element $e \in G$ such that $a * e = a$ and $e * a = a$ for every element $a \in G$ 
\end{axiom}
\begin{axiom}
There is an element $a^{-1} \in G$ for every element $a \in G$, such that $a * a^{-1} = e$ and $a^{-1} * a = e$ 
\end{axiom}
If group is also commutative we call it \textit{Abelian}. Operation is not necessarily multiplication (addition, modulo addition, boolean operation are some examples).

\section{Elementary Properties of Groups}
There is \textbf{exactly one} identity element $e$ and \textbf{exactly one} inverse element $a{-1}$ in every group.
\begin{theorem}{}
If $G$ is a group and $a, b, c \in G$ then
\begin{itemize}
\item $ab = ac$ implies $b = c$ and
\item $ba = ca$ implies $b = c$ 
\end{itemize}
\end{theorem}
\begin{theorem}{}
If $G$ is a group and $a, b \in G$ then
\begin{itemize}
\item $ab = e$ implies $a = b^{-1}$ and $b = a^{-1}$  
\end{itemize}
\end{theorem}
\begin{theorem}{}
If $G$ is a group and $a, b \in G$ then
\begin{itemize}
\item $(ab)^{-1} = b^{-1}a^{-1}$ and
\item $(a^{-1})^{-1} = a$  
\end{itemize}
\end{theorem}
More generaly 
\begin{equation*}
(a_1a_2 \cdots a_n)^{-1} = a_{n}^{-1} \cdots a_{2}^{-1}a_{1}^{-1}
\end{equation*}
If $G$ is finite, number of elements in $G$ is called \textit{order of} $G$ and denoted $|G|$.
Direct product of two groups $G$ and $H$ is new group $G \times H$ defined as follows: $G \times H$ consists of all the ordered pairs $(x, y)$ where $x \in G$ and $y \in H$
\begin{equation*}
   G \times H = \Big\{(x, y): x \in G \text{ and } y \in H\Big\}
\end{equation*}
Let $G$ be a group with $a, b \in G$. For any positive integer $n$, we define $a^n$ by
\begin{equation*}
a^n = \underbrace{aaa \cdots a}_\text{n factors}
\end{equation*}
If $a = z^n$ for some $z \in G$ we say that a has $n$th root in $G$.
\section{Subgroups}
\textbf{Subgroup} $S$ of group $\langle G, * \rangle$ is a nonempty subset of $G$ closed with respect to group $G$ operation and closed with respect to inverses.
\\
Subgroup of $G$ \textit{generated by elements} $a_1, \dots ,a_n$ is a group that contains all possible products of these elements and their inverses.
\\
If subgroup is generated by single element $a \in G$ (\textit{generator}), it is called a \textit{cyclic subgroup of} $G$, designated by $\langle a \rangle$.
\\
Finite groups can be represented using \textit{Cayley diagrams}. There is one point for every element of the group and arrows between points represent the result of multiplying by a generator.
\section{Functions}
\begin{definition}
        A function $f : A \to B$ is called \textbf{injective} if each element of $B$ is the image of no more than one element of $A$, that is $f(x_1) = f(x_2) \implies x_1 = x_2$.
\end{definition}
\begin{definition}
        A function $f : A \to B$ is called \textbf{surjective} if each element of $B$ is the image of at least one element of $A$, that is $B$ is the \textit{range} of $f$.
\end{definition}
\begin{definition}
    A function $f: A \to B$ is called \textbf{bijective} if it is both injective and surjective.
\end{definition}
\begin{definition}{}
        \textnormal{Let $f: A \to B $ and  $g: B \to A$ be functions. The \textbf{composite function} denoted by $g \circ f$ is a function from \textit{A} to \textit{C} as follows:
        $[g \circ f](x) = g(f(x))$ }
\end{definition}
Composition of functions is associative operation.
\\
A function $f: A \to B$ has an inerse iff bijective, that is inverse $f^{-1}: B \to A$ is a bijective function
\section{Groups of Permutations}
A \textbf{permutation} of a set $A$ is a bijective function from $A$ to $A$ - a one-to-one correspondence between $A$ and itself. 
\\
Composition of bijective functions is bijective so the composition of any two permutations of $A$ gives a permutation of $A$.
\\
In context of groups, composition can be thought of as an operation on a set of all the permutations of $A$. To fulfill other requirements we need an identity and inverse.
\\
Identity function $\varepsilon$ on $A$ is the function $x \mapsto x$ which carries every element of A to itself $\varepsilon(x) = x$, $\forall x \in A$.
\\
Inverse of any permutation of $A$ is a permutation of $A$, that is $f^{-1} \circ f = \varepsilon$ and $f \circ f^{-1}= \varepsilon$.
\begin{definition}{}
        \textnormal{The set of all permutations of $A$ with operation $\circ$ of composition is a group - called \textbf{symmetric group} on $A$, denoted $S_A$.}
\end{definition}
For any positive integer $n$, symmetric group $S_n$ on the set $\{1, 2, 3, \dots, n\}$ is called the symmetric group on $n$ elements.
\section{Permutations of a Finite Set}
\textit{Cycle} $(a_1a_2 \dots a_s)$ is a permutation which shifts first $s$ elements of set ${1,2,\dots ,n}$ cyclically by one while leaving other $n-s$ elements unchanged. Integer s is called \textit{length} of the cycle. The composite of two cycles is called their \textit{product}. If two cycles don't have common elements they are \textit{disjoint} and commutative.
\begin{theorem}{}
Every permutation is either the identity, a single cycle or a product of disjoint cycles.
\end{theorem}
\textit{Transposition} is a cycle of length 2. Every permutation can be written as the product of transpositions. Dissociation in transpositions is not unique, but final number of product is always either \textit{even} or \textit{odd} for the same permutation.
\begin{theorem}{}
Number of transpositions in identity permutation $\varepsilon$ is always even.
\end{theorem}
If $\alpha$ is any cycle and $\pi$ any permutation $\pi\alpha\pi^{-1}$ is called a \textit{conjugate} of $\alpha$.
\section{Isomorphism}
\begin{definition}{}
    An \textbf{isomorphism} from group $G1$ to group $G2$ is a bijective function $f: G_1 \to G_2$ with property that for any two elements $a, b \in G_1$
        \begin{equation}
                f(ab) = f(a)f(b)
        \end{equation}
\end{definition}
Informally, groups $G1$ and $G2$ are isomorphic if they have same elements with different names.
\begin{definition}{}
    \textbf{Cayley's Theorem}: Every group is isomorphic to a group of permutations.
\end{definition}
Isomorphism is an equivalence relation among groups.
\section{Order of Group Elements}
\begin{theorem}{Law of Exponents.}
    If $G$ is a group and $a \in G$ the following identities hold for all integers $m$ and $n$:
        \begin{equation*}\tag{i}
            a^ma^n = a^{m+n}
        \end{equation*}
        \begin{equation*}\tag{ii}
            (a^m)^n = a^{mn}
        \end{equation*}
        \begin{equation*}\tag{iii}
            a^{-n} = (a^{-1})^n = (a^n)^{-1}
        \end{equation*}
\end{theorem}
\begin{theorem}{Division Algorithm.}
        If $m$ and $n$ are integers and $n$ is positive, there exists unique integers $q$ (\textit{quotient}) and $r$ (\textit{remainder}) such that 
    \begin{equation*}
            m = nq + r \text{ and } 0 \leq r < n
    \end{equation*}
\end{theorem}
If there exists a nonzero integer $m$ such that $a^m=e$ then the \textbf{order} of the element $a$ is defined to be the \textbf{least positive} integer $n$ such that $a^n=e$. If such $m$ does not exist, $a$ has order infinity.
\begin{theorem}{}
        If the order of $a$ is $n$ there are exactly $n$ different powers of $a$.
\end{theorem}
\begin{theorem}{}
        If the order of $a$ is infinte all powers of $a$ are different.
\end{theorem}
\section{Cyclic Groups}
\begin{theorem}{Isomorphism of Cyclic Groups}
\begin{itemize}
        \item For every positive integer $n$, every cyclic group of order $n$ is ismorphic to $\mathbb{Z}_n$. Any two cyclic groups of order $n$ are isomporhic to each other. 
       \item Every cyclic group of order infinity is isomporhic to $\mathbb{Z}$
\end{itemize} 
\end{theorem}
\begin{theorem}{}
        Every subgroup of a cyclic group is cyclic.
\end{theorem}
\section{Partitions and Equivalence Relations}
Partition of a set $A$ is a familiy $\{ A_i : i \in I \}$ of nonempty subsets of A which are mutually disjoint and whose union is all of $A$. If any two subsets $A_i$ and $A_j$ have a common element $x$ then $A_i = A_j$. Every element of $A$ lies in one of the subsets.
\\
\textit{A relation} on a set $A$ is a statement which is either true or false for each ordered pair $(x, y)$ of elements of $A$.
Equivalence relation on a set $A$ is a relation $\sim$ which is 
\begin{itemize}
    \item \textit{Reflexive} $x \sim x$ , $\forall x \in A$
    \item \textit{Symmetric} if $x \sim y$ , then $y \sim x$
    \item \textit{Transitive} if $x \sim y$ and $y \sim z$, then $x \sim z$
\end{itemize}
The set of all elements equivalent to $x$ is \textit{equivalence class} of $x$
\begin{equation*}
        [x] = \{y \in A : y \sim x\}
\end{equation*}
\begin{lemma}{}
        If $x \sim y$ then $[x] = [y]$
\end{lemma}
\begin{theorem}
        If $\sim$ is an equivalence relation on $A$ the family of all the equivalence classes is a partition of $A$.
\end{theorem}
\section{Counting Cosets}
If $H$ is a subgroup of $G$, for any element $a$ in $G$ the symbol $aH$ denotes the set of all products $ah, \forall h \in H$ where $a$ is fixed. This is a \textbf{left coset} of $H$ in $G$. Similarly $Ha$ is \textbf{right coset} of $H$ in $G$.
\begin{theorem}{}
    The family of all cosets $Ha$ as $a$ ranges over $G$ is a partition of $G$.
\end{theorem}
For $H$ subgroup of $G$, all the cosets of $H$ have the same number of elements.
\begin{theorem}{}
    If $Ha$ is any coset of $H$ there is a one-to-one correspondence from $H$ to $Ha$.
\end{theorem}
\begin{theorem}{Lagrange's theorem}
        Let $H$ be any subgroup of a finite group $G$. The order of $G$ is a multiple of order of $H$.
\end{theorem}
\begin{theorem}{}
    If $G$ is a group with a prime number $p$ of elements, then $G$ is a cyclic group. Any element $a \neq e$ in $G$ is a generator of $G$.
\end{theorem}
\begin{theorem}{}
        The order of any element of a finite group divides the order of group.
\end{theorem}
If $G$ is a finite group and $p$ is a prime divisor of $|G|$ then $G$ has an element of order $p$.
\section{Homomorphisms}
If $G$ and $H$ are groups, a textbf{homomorphism} from $G$ to $H$ is a function $f: G \to H$ such that for any two elements $a,b \in G$ 
\begin{equation*}
    f(ab) = f(a)f(b)
\end{equation*}
\begin{theorem}{}
        Let $G, H$ be groups and $f: G \to H$ a homomorphism. Then
        \begin{itemize}
            \item $f(e) = e$ and
            \item $f(a^{-1}) = [f(a)]^{-1}$ for every element $a \in G$
        \end{itemize}
\end{theorem}
If $a$ is an element of group $G$, a \textit{conjugate of} $a$ is any element of the form $xax^{-1}, x \in G$.
\begin{definition}{}
$H$ is called normal subgroup of $G$ if it is closed with respect to conjugates if
        \begin{equation*}
                \text{for any } a \in H \text{ and } x \in G xax^{-1} \in H
        \end{equation*}
\end{definition}
\begin{definition}{}
        The \textbf{kernel} of $f: G \to H$ is the set $K$ of all the elements of $G$ which are carried by $f$ onto the neutral element of $H$.
\end{definition}
\begin{theorem}{}
        Let $f: G \to H$ be a homomorphism. 
        \begin{itemize}
            \item The kernel of $f$ is a normal subgroup of $G$ and
            \item The range of $f$ is a subgroup of $H$
        \end{itemize}
\end{theorem}
\section{Quotient Groups}
\begin{theorem}{}
    If $H$ is a normal subgroup of $G$ then $aH = Ha$ for every $a \in G$.
\end{theorem}
Coset multiplication: the coset of $a$ multiplied by the coset of $b$ is defined to be coset of $ab$.
\begin{theorem}{}
    Let $H$ be a normal subgroup of $G$. If $Ha = Hc$ and $Hb = Hd$ then $H(ab) = H(cd)$.
\end{theorem}
\begin{theorem}{}
    Set of all the cosets of $H$, denoted $G/H$ with coset multiplication is a group.
\end{theorem}
\begin{theorem}{}
    $G/H$ is a homomorphic image of G.
\end{theorem}
\begin{theorem}{}
        For $H$, a subgroup of $G$
        \begin{equation*}\tag{a}
            Ha = Hb \iff ab^{-1} \in H \text{ and }
        \end{equation*}
        \begin{equation*}\tag{b}
            Ha = H \iff a \in H
        \end{equation*}
\end{theorem}
\section{The Fundamental Homomorphism Theorem}
\begin{theorem}{}
    Let $f: G \to H$ be a homomorphism with kernel K. Then $f(a) = f(b)$ iff $Ka = Kb$.
\end{theorem}
\begin{theorem}{}
        Let $f: G \to H$ be a homomorphism of $G$ \textbf{onto} $H$. If $K$ is kernel of $f$, then
        $H \cong G/K$
\end{theorem}
\section{Rings: Definitions and Elementary Properties}
\begin{definition}{}
    Ring is a set $A$ with operations called addition and multiplication which satisfy the following axioms:
        \begin{enumerate}
            \item $A$ with addition alone is an abelian group.
            \item Multiplication is associative
            \item Multiplication is distributive over addition
        \end{enumerate}
\end{definition}
\begin{theorem}{}
    Let $a$ and $b$ be any elements of a ring $A$.
        \renewcommand{\labelenumiii}{\theenumi}
        \begin{enumerate}
            \item $a0 = 0$ and $0a = 0$
            \item a(-b) = -(ab) and (-a)b = -(ab)
            \item (-a)(-b) = ab
        \end{enumerate}
\end{theorem}
\begin{itemize}
    \item \textit{Commutative} ring has commutative multiplication too.
    \item \textit{Ring with unity} has neutral element for multipilcation.
    \item \textit{Field} is a commutative ring with unity in which every nonzero element is invertible.
\end{itemize}
\begin{definition}{}
        In any ring, a nonzero element $a$ is called a \textbf{divisor of zero} if there is a nonzero element $b$ in the ring such that the product $ab$ or $ba$ is equal to \textit{zero}. If the product of two elements in the ring is equal zeor at least one of the factors is zero (ring doesn't divisors of zero).
\end{definition}
\begin{definition}{}
    A ring is said to have the \textbf{cancelation property} if
        \begin{equation*}
            ab = bc \text{ or } ba = ca \text{ implies } b = c
        \end{equation*}
    for any elements $a, b, c$ in the ring if $a \neq 0$.
\end{definition}
\begin{theorem}{}
    A ring has cancelation property \textit{iff} it has no divisors of zero.
\end{theorem}
\section{Ideals and Homomorphisms}
If a nonempty subset $B \subseteq A$ is closed with respect to addition, multiplication and negatives, then $B$ with the operations of $A$ is a ring.
\\
$B$ is a subring of $A$ \textit{iff} $B$ is closed with respect to subtraction and multiplication.
\\
A nonempty subset $B$ of a ring $A$ is called an \textbf{ideal} of $A$ if $B$ is closed with respect to addition and negatives and $B$ \textit{absorbs products} (whenever we multiply an element in $B$ by an element in $A$, which is not necessarily in $B$, their product is always in B) in $A$.
\\
The \textit{principal ideal generated by $a$} $\langle a \rangle$ is the set of all mutiples of a fixed element $a$ by all the elements in a commutative ring with unity.
\\
$B$ is an ideal of $A$ \textit{if and only if} $B$ is closed with respect to subtraction and $B$ absorbs products in $A$.
\begin{definition}{}
    A \textbf{homomorphism} from a ring $A$ to a ring $B$ is a function $f: A \to B$ satisfying the identities 
        \begin{equation*}
            f(x_1+x_2)=f(x_1)+f(x_2)
        \end{equation*}
        and
        \begin{equation*}
            f(x_1x_2)=f(x_1)f(x_2)
        \end{equation*}
\end{definition}
The \textit{kernel of f} is the set of all the elements of $A$ which are carried by $f$ onto the \textit{zero} element of $B$.
\begin{equation*}
    K = \{x\in A: f(x)=0\}
\end{equation*}
The kernel of $f$ is an ideal of $A$.
\section{Quotient Rings}
\begin{definition}{}
        Let $A$ be a ring, and $J$ an ideal of $A$. For any $a \in A$ the \textbf{coset of J in A} is the set of all sums $j+a$ as $a$ remains fixed and $j$ ranges over $J$:
\begin{equation*}
        J + a = \{j + a: j \in J \}
\end{equation*}
\end{definition}
\begin{theorem}{}
    Let $J$ be an ideal of $A$. If $J+a=J+c$ and $J+b=J+d$, then
    \begin{enumerate}
            \item $J+(a+b)=J+(c+d)$
            \item $J+ab=J+cd$
    \end{enumerate}
\end{theorem}
\begin{theorem}{}
    The set of \textit{all} cosets of $J$ in $A$ 
        \begin{equation*}
            A/J = \{J+a, J+b, J+c, ...\}
        \end{equation*}
        with coset addition and multiplication is a ring. It is called \textit{quotient} ring.
\end{theorem}
\begin{theorem}{}
    $A/J$ is a homomorphic image of $A$.
\end{theorem}
\begin{theorem}{}
        Let $f: A \to B$ be a homomorphism from a ring $A$ \textbf{onto} a ring $B$ and let $K$ be the kernel of $f$. Then $B \cong A/K$.
\end{theorem}
\section{Integral Domains}
Integral domain is a commutative ring with unity having the cancelation property.

\begin{equation}
        \text{if } a \neq 0 \text{ and } ab = ac \text{ then } b = c
\end{equation}
\begin{equation}
        \text{if } ab = 0 \text{ then } a = 0 \text{ or } b = 0
\end{equation}
If $A$ is the ring with unity, the \textbf{characteristic of $A$} is the least positive integer $n$ such that
\begin{equation*}
\underbrace{1 + 1 + \cdots + 1 = 0}_\text{n times}
\end{equation*}
This is \textit{additive order} of $a \in A$
\begin{theorem}{}
    All the nonzero elements in an integral domain have the same additive order.
\end{theorem}
\begin{theorem}{}
    In an integral domain with nonzero characteristic, the characteristic is a prime number.
\end{theorem}
\begin{theorem}{}
    In any integral domain of characteristic \textbf{p}
        \begin{equation*}
                (a+b)^p = a^p + b^p \;\; \forall a,b
        \end{equation*}
\end{theorem}
\begin{theorem}{}
    Every finite integral domain is a field
\end{theorem}
\end{document}
