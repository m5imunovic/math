\documentclass[12pt]{article}
\usepackage{amsmath}
\usepackage{amsfonts}
\usepackage{amssymb}
\usepackage[margin=1in]{geometry}
\title{Notes on Abstract Algebra by C.C. Pinter}
\author{M. Simunovic}
\begin{document}
\maketitle

\section*{Notes to me}
\begin{itemize}
        \item Set can be anything, not limited to numbers.
        \item Remember that sets might contain infinite number of members.
\end{itemize}
\section*{Operations}
An operation $*$ on $A$ is a rule which assigns to each ordered pair $(a,b)$ of elements of A exactly one element $a * b$ in A.
\begin{equation}
        \textnormal{$a * b$ is defined for \textbf{every} ordered pair $(a, b)$ of elements of $A$.}
\end{equation}
\begin{equation}
        \textnormal{$a * b$ must be \textbf{uniquely} defined.}
\end{equation}
\begin{equation}
        \textnormal{If $a$ and $b$ are in $A$, $a * b$ must be in $A$.}
\end{equation}
\begin{equation}
        \textnormal{Operation is \textit{commutative} if $a * b = b * a$ for any $a$ and $b$ in $A$.}
\end{equation}
\begin{equation}
        \textnormal{Operation is \textit{associative} if $(a * b) * c = a * (b * c)$ for any three elements $a$, $b$ and $c$ in $A$.}
\end{equation}
\begin{equation}
        \textnormal{Element $e \in A$ with the property $e * a = a$ and $a * e = a$ for every element $a \in A$ is called identity element.}
\end{equation}

\begin{equation}
        \textnormal{Element $x \in A$ with the property $x * a = e$ and $a * x = e$ for any element $a \in A$ is called inverse of $a$.}
\end{equation}
\end{document}
