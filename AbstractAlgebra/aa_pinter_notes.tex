\documentclass[a4paper,12pt]{article}
\usepackage{amsmath}
\usepackage{amsthm}
\usepackage{amsfonts}
\usepackage{amssymb}
\usepackage[margin=1in]{geometry}
%\usepackage{showframe}
\title{Notes on Abstract Algebra by C.C. Pinter}
\author{M. Simunovic}

\theoremstyle{definition}
\newtheorem{definition}{}[section]
\theoremstyle{axiom}
\newtheorem{axiom}{G}[]
\theoremstyle{theorem}
\newtheorem{theorem}{Theorem}[section]
\begin{document}
\maketitle

\section{Notes to me}
\begin{itemize}
        \item Set can be anything, not limited to numbers.
        \item Always remember that sets might contain infinite number of members.
\end{itemize}
\section{Operations}
An operation $*$ on $A$ is a rule which assigns to each ordered pair $(a,b)$ of elements of A exactly one element $a * b$ in A.
\begin{definition}{}
        \textnormal{$a * b$ is defined for \textbf{every} ordered pair $(a, b)$ of elements of $A$.}
\end{definition}
\begin{definition}{}
        \textnormal{$a * b$ must be \textbf{uniquely} defined.}
\end{definition}
\begin{definition}{}
        \textnormal{If $a$ and $b$ are in $A$, $a * b$ must be in $A$.}
\end{definition}
\begin{definition}{}
        \textnormal{Operation is \textit{commutative} if $a * b = b * a$ for any $a$ and $b$ in $A$.}
\end{definition}
\begin{definition}{}
        \textnormal{Operation is \textit{associative} if $(a * b) * c = a * (b * c)$ for any three elements $a$, $b$ and $c$ in $A$.}
\end{definition}
\begin{definition}{}
        Element $e \in A$ with the property $e * a = a$ and $a * e = a$ for every element $a \in A$  is called identity element.
\end{definition}
\begin{definition}{}
        \textnormal{Element $x \in A$ with the property $x * a = e$ and $a * x = e$ for any element $a \in A$ is called inverse of $a$.}
\end{definition}

\section{The Definition of Groups}
A \textbf{group} $\langle G, * \rangle$ is \textit{a set $G$ with an operation $*$} that satisfies the axioms:
\begin{axiom}
$*$ is associative
\end{axiom}
\begin{axiom}
There is an element $e \in G$ such that $a * e = a$ and $e * a = a$ for every element $a \in G$ 
\end{axiom}
\begin{axiom}
There is an element $a^{-1} \in G$ for every element $a \in G$, such that $a * a^{-1} = e$ and $a^{-1} * a = e$ 
\end{axiom}
If group is also commutative we call it \textit{Abelian}. Operation is not necessarily multiplication (addition, modulo addition, boolean operation are some examples).

\section{Elementary Properties of Groups}
There is \textbf{exactly one} identity element $e$ and \textbf{exactly one} inverse element $a{-1}$ in every group.
\begin{theorem}{}
If $G$ is a group and $a, b, c \in G$ then
\begin{itemize}
\item $ab = ac$ implies $b = c$ and
\item $ba = ca$ implies $b = c$ 
\end{itemize}
\end{theorem}
\begin{theorem}{}
If $G$ is a group and $a, b \in G$ then
\begin{itemize}
\item $ab = e$ implies $a = b^{-1}$ and $b = a^{-1}$  
\end{itemize}
\end{theorem}
\begin{theorem}{}
If $G$ is a group and $a, b \in G$ then
\begin{itemize}
\item $(ab)^{-1} = b^{-1}a^{-1}$ and
\item $(a^{-1})^{-1} = a$  
\end{itemize}
\end{theorem}
More generaly 
\begin{equation*}
(a_1a_2 \cdots a_n)^{-1} = a_{n}^{-1} \cdots a_{2}^{-1}a_{1}^{-1}
\end{equation*}
If $G$ is finite, number of elements in $G$ is called \textit{order of} $G$ and denoted $|G|$.
Direct product of two groups $G$ and $H$ is new group $G \times H$ defined as follows: $G \times H$ consists of all the ordered pairs $(x, y)$ where $x \in G$ and $y \in H$
\begin{equation*}
   G \times H = \Big\{(x, y): x \in G \text{ and } y \in H\Big\}
\end{equation*}
Let $G$ be a group with $a, b \in G$. For any positive integer $n$, we define $a^n$ by
\begin{equation*}
a^n = \underbrace{aaa \cdots a}_\text{n factors}
\end{equation*}
If $a = z^n$ for some $z \in G$ we say that a has $n$th root in $G$.
\section{Subgroups}
\textbf{Subgroup} $S$ of group $\langle G, * \rangle$ is a nonempty subset of $G$ closed with respect to group $G$ operation and closed with respect to inverses.
\\
Subgroup of $G$ \textit{generated by elements} $a_1, \dots ,a_n$ is a group that contains all possible products of these elements and their inverses.
\\
If subgroup is generated by single element $a \in G$ (\textit{generator}), it is called a \textit{cyclic subgroup of} $G$, designated by $\langle a \rangle$.
\\
Finite groups can be represented using \textit{Cayley diagrams}. There is one point for every element of the group and arrows between points represent the result of multiplying by a generator.
\section{Functions}
\begin{definition}
        A function $f : A \to B$ is called \textbf{injective} if each element of $B$ is the image of no more than one element of $A$, that is $f(x_1) = f(x_2) \implies x_1 = x_2$.
\end{definition}
\begin{definition}
        A function $f : A \to B$ is called \textbf{surjective} if each element of $B$ is the image of at least one element of $A$, that is $B$ is the \textit{range} of $f$.
\end{definition}
\begin{definition}
    A function $f: A \to B$ is called \textbf{bijective} if it is both injective and surjective.
\end{definition}
\begin{definition}{}
        \textnormal{Let $f: A \to B $ and  $g: B \to A$ be functions. The \textbf{composite function} denoted by $g \circ f$ is a function from \textit{A} to \textit{C} as follows:
        $[g \circ f](x) = g(f(x))$ }
\end{definition}
Composition of functions is associative operation.
\\
A function $f: A \to B$ has an inerse iff bijective, that is inverse $f^{-1}: B \to A$ is a bijective function
\section{Groups of Permutations}
A \textbf{permutation} of a set $A$ is a bijective function from $A$ to $A$ - a one-to-one correspondence between $A$ and itself. 
\\
Composition of bijective functions is bijective so the composition of any two permutations of $A$ gives a permutation of $A$.
\\
In context of groups, composition can be thought of as an operation on a set of all the permutations of $A$. To fulfill other requirements we need an identity and inverse.
\\
Identity function $\varepsilon$ on $A$ is the function $x \mapsto x$ which carries every element of A to itself $\varepsilon(x) = x$, $\forall x \in A$.
\\
Inverse of any permutation of $A$ is a permutation of $A$, that is $f^{-1} \circ f = \varepsilon$ and $f \circ f^{-1}= \varepsilon$.
\begin{definition}{}
        \textnormal{The set of all permutations of $A$ with operation $\circ$ of composition is a group - called \textbf{symmetric group} on $A$, denoted $S_A$.}
\end{definition}
For any positive integer $n$, symmetric group $S_n$ on the set $\{1, 2, 3, \dots, n\}$ is called the symmetric group on $n$ elements.
\section{Permutations of a Finite Set}
\textit{Cycle} $(a_1a_2 \dots a_s)$ is a permutation which shifts first $s$ elements of set ${1,2,\dots ,n}$ cyclically by one while leaving other $n-s$ elements unchanged. Integer s is called \textit{length} of the cycle. The composite of two cycles is called their \textit{product}. If two cycles don't have common elements they are \textit{disjoint} and commutative.
\begin{theorem}{}
Every permutation is either the identity, a single cycle or a product of disjoint cycles.
\end{theorem}
\textit{Transposition} is a cycle of length 2. Every permutation can be written as the product of transpositions. Dissociation in transpositions is not unique, but final number of product is always either \textit{even} or \textit{odd} for the same permutation.
\begin{theorem}{}
Number of transpositions in identity permutation $\varepsilon$ is always even.
\end{theorem}
If $\alpha$ is any cycle and $\pi$ any permutation $\pi\alpha\pi^{-1}$ is called a \textit{conjugate} of $\alpha$.
\section{Isomorphism}
\begin{definition}{}
    An \textbf{isomorphism} from group $G1$ to group $G2$ is a bijective function $f: G_1 \to G_2$ with property that for any two elements $a, b \in G_1$
        \begin{equation}
                f(ab) = f(a)f(b)
        \end{equation}
\end{definition}
Informally, groups $G1$ and $G2$ are isomorphic if they have same elements with different names.
\begin{definition}{}
    \textbf{Cayley's Theorem}: Every group is isomorphic to a group of permutations.
\end{definition}
\end{document}
